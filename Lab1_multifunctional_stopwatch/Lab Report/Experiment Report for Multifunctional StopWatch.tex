\documentclass[UTF8]{ctexart}
\usepackage{listings}
\title{多功能秒表实验报告}
\author{陈子轩 516030910545}
\date{\today}

\begin{document}
\maketitle
\section{实验目的}
\noindent1. 初步掌握利用Verilog 硬件描述语言进行逻辑功能设计的原理和方法。
\newline 2. 理解和掌握运用大规模可编程逻辑器件进行逻辑设计的原理和方法。
\newline 3. 理解硬件实现方法中的并行性,联系软件实现方法中的并发性。
\newline 4. 理解硬件和软件是相辅相成、并在设计和应用方法上的优势互补的特点。
\newline 5. 本实验学习积累的Verilog 硬件描述语言和对 FPGA/CPLD的编程操作,是进行后续《计算机组成原理》部分课程实验,设计实现计算机逻辑的基础。

\section{实验内容和任务}
\noindent 1. 运用Verilog 硬件描述语言,基于 DE1-SOC实验板,设计实现一个具有较多 功能的计时秒表。
\newline 2. 要求将 8 个数码管设计为具有“时:分:秒:毫秒”显示,按键的基本控制 动作有 3 个:“计时复位”、“计数/暂停”、“显示暂停/显示继续”。功能能够 满足马拉松或长跑运动员的计时需要。
\newline 3. 利用示波器观察按键的抖动,设计按键电路的消抖方法。
\newline 4. 在实验报告中详细报告自己的设计过程、步骤及Verilog 代码。

\section{实验仪器}
Altera – DE1-SOC实验板1套,示波器1台,数字万用表1台。
\section{实验设计思路}
\subsection{秒表设计思路}
\paragraph{1表示工作状态}
创建三个寄存器变量counter\_work, display\_work, 和reset\_work来表示此时秒表的状态如果counter\_work为1说明此时计时器是工作的;如果display\_work为1说明此时显示器是工作的;如果reset\_work说明此时需要将所有数码管置零。
\paragraph{2判定工作状态}
接下来是如何判断这三个寄存器的工作状态。我的方法是设置三个寄存器保存上一个时钟上跳沿三个按键key\_reset, key\_start\_pause, key\_display\_stop的状态。如果上一个状态按键没有按下(状态为1),该时钟上跳沿按键按下(状态为0),说明是首次按下,我就将上述三个寄存器变量(counter\_work, display\_work, 和reset\_work)的值反转。
\paragraph{3实现工作状态}
最后就是对应个状态的功能实现了,对于“计数/暂停”功能,设置6个计时器分别对应6个数码管。设置一个计数器。如果计数器达到500000,由于时钟是50MHz的,所以说明距上次计时器状态更新已经过了10ms,则依次进位更新各计时器。对于“显示暂停/显示继续”功能,则简单的将计时器的值显示在数码管上。对于“计时复位”功能,则简单的将各计时器置零即可。
\subsection{防抖设计思路}
上述设计思路并没有考虑按键抖动的情况,所以接下来还需针对抖动进行设计。我的设计思路是创建三个按键状态时间计数器,它记录了假定本次按键按下距离上次判定按键按下的时间。如果这个时间小于某一个值,我们就判定这次按键按下的原因是因为抖动,所以实际上按键并未按下。所以我们的设计只需对4.1中的第二步做修改即可。

\section{部分实验代码}

\begin{verbatim}
always @ (posedge clk) //每一个时钟上升沿开始触发下面的逻辑
    begin
    //消抖寄存器更新
    counter_reset = #1 counter_reset + 1;
    counter_start = #1 counter_start + 1;
    counter_display = #1 counter_display + 1;


    //判断是否触发开关
    if(key_start_pause == 0 && previous_key_start_pause == 1 
        && counter_start >= DELAY_TIME)
        //如果开关从1变0,并且在200ms时间之内没有触发开关
        begin
        counter_start = #1 0;
        counter_work = #1 counter_work + 1;
        end
    if(key_display_stop == 0 && previous_key_display_stop == 1 
        && counter_display >= DELAY_TIME)
        begin
        display_work = #1 display_work + 1;
        counter_display = #1 0;
        end
    if(key_reset == 0 && previous_key_reset == 1 && counter_reset >= DELAY_TIME)
        begin
        reset_work = #1 1;
        counter_reset = #1 0;
        end



    //触发了“暂停计时”开关
    if(counter_work)
        begin
        counter_50M = #1 counter_50M + 1;
        if(counter_50M == 500000)
            begin
            counter_50M = #1 0;
            msecond_counter_low = #1 msecond_counter_low + 1;
            if(msecond_counter_low == 9)
                begin
                msecond_counter_low = #1 0;
                msecond_counter_high = #1 msecond_counter_high + 1;
                if(msecond_counter_high == 9)
                    begin
                    msecond_counter_high = #1 0;
                    second_counter_low = #1 second_counter_low + 1;
                    if(second_counter_low == 9)
                        begin
                        second_counter_low = #1 0;
                        second_counter_high = #1 second_counter_high + 1;
                        if(second_counter_high == 5)
                            begin
                            second_counter_high = #1 0;
                            minute_counter_low = #1 minute_counter_low + 1;
                            if(minute_counter_low == 9)
                                begin
                                minute_counter_low = #1 0;
                                minute_counter_high = #1 minute_counter_high + 1;
                                if(minute_counter_high == 5)
                                    begin
                                    minute_counter_high = #1 0;
                                    end
                                end
                            end
                        end
                    end
                end
            end
        end

    //触发了“系统复位”开关
    if(reset_work == 1)
        begin
        minute_counter_high = #1 0;
        minute_counter_low =#1 0;
        msecond_counter_high = #1 0;
        msecond_counter_low = #1 0;
        second_counter_high =#1  0;
        second_counter_low = #1 0;
        reset_work = #1 0;
        end

    //触发了“暂停显示”开关
    if(display_work == 0)
        begin
        minute_display_high = #1 minute_counter_high;
        minute_display_low = #1 minute_counter_low;
        second_display_high = #1 second_counter_high;
        second_display_low = #1 second_counter_low;
        msecond_display_high = #1 msecond_counter_high;
        msecond_display_low = #1 msecond_counter_low;
        end

    //更新上周期的的寄存器状态的记录
    previous_key_display_stop = #1 key_display_stop;
    previous_key_start_pause = #1 key_start_pause;
    previous_key_reset = #1 key_reset;
    end
\end{verbatim}

\section{实验感想}
本次实验我学到了Quartus平台的使用方法和verilog硬件编程语言的语法以及如何运用可编程逻辑器件进行逻辑设计的方法。后者是我以前从没接触到的,所以一开始上手感觉比较新奇,我发现我们在硬件编程的所有逻辑都可以用逻辑器件进行表达,所以这比较强调我们软硬件的综合能力。
\end{document}

